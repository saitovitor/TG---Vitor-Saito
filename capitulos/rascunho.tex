Substituindo $(u_{i}, v_{i}, w_{i})$, para $i = {1, 2, 3, 4}$, na equação \ref{eqn:interpolation}, tem-se:

\begin{equation} \label{eq:matrixdisplacement}
    \begin{Bmatrix}
        u_{1} \\
        v_{1} \\
        w_{1} \\
        u_{2} \\
        v_{2} \\
        w_{2} \\
        u_{3} \\
        v_{3} \\
        w_{3} \\
        u_{4} \\
        v_{4} \\
        w_{4} \\
    \end{Bmatrix}
    = 
    \begin{bmatrix}
    1 & x_{1} & y_{1} & z_{1} & 0 & 0 & 0 & 0 & 0 & 0 & 0 & 0 \\
    0 & 0 & 0 & 0 &  1 & x_{1} & y_{1} & z_{1} & 0 & 0 & 0 & 0 \\
    0 & 0 & 0 & 0 & 0 & 0 & 0 & 0 & 1 & x_{1} & y_{1} & z_{1} \\

    1 & x_{2} & y_{2} & z_{2} & 0 & 0 & 0 & 0 & 0 & 0 & 0 & 0 \\
    0 & 0 & 0 & 0 &  1 & x_{2} & y_{2} & z_{2} & 0 & 0 & 0 & 0 \\
    0 & 0 & 0 & 0 & 0 & 0 & 0 & 0 & 1 & x_{2} & y_{2} & z_{2} \\
    
    1 & x_{3} & y_{3} & z_{3} & 0 & 0 & 0 & 0 & 0 & 0 & 0 & 0 \\
    0 & 0 & 0 & 0 &  1 & x_{3} & y_{3} & z_{3} & 0 & 0 & 0 & 0 \\
    0 & 0 & 0 & 0 & 0 & 0 & 0 & 0 & 1 & x_{3} & y_{3} & z_{3} \\
    
    1 & x_{4} & y_{4} & z_{4} & 0 & 0 & 0 & 0 & 0 & 0 & 0 & 0 \\
    0 & 0 & 0 & 0 &  1 & x_{4} & y_{4} & z_{4} & 0 & 0 & 0 & 0 \\
    0 & 0 & 0 & 0 & 0 & 0 & 0 & 0 & 1 & x_{4} & y_{4} & z_{4} \\

    \end{bmatrix}
    \begin{Bmatrix}
        \alpha_{1} \\
        \alpha_{2} \\
        \alpha_{3} \\
        \alpha_{4} \\
        \alpha_{5} \\
        \alpha_{6} \\
        \alpha_{7} \\
        \alpha_{8} \\
        \alpha_{9} \\
        \alpha_{10} \\
        \alpha_{11} \\
        \alpha_{12} \\
    \end{Bmatrix}
    ,
\end{equation}

\begin{equation} \label{eq:simplifieddisplacement}
    \pmb{u_{e}} = \pmb{A}\alpha,
\end{equation}

A deformação de engenharia $\epsilon$ é definida como:

\begin{equation} \label{eq:deformaçãoengenharia}
    \epsilon = \frac{l}{l_{0}},
\end{equation}

na qual, $l$ e $l_{0}$ são, respectivamente, o comprimento de um elemento e seu deslocamento.

Aplicando \ref{eq:deformaçãoengenharia} para os eixos $xyz$, tem-se
\begin{equation}
    \begin{gathered}
    &\epsilon_{x} = \frac{\partial u}{\partial x} \\
    &\epsilon_{y} = \frac{\partial v}{\partial y} \\
    &\epsilon_{z} = \frac{\partial w}{\partial z}
    \end{gathered}
\end{equation}

A deformação angular $\gamma$ é definida como:
\begin{equation} \label{eq:angulardef}
    \begin{gathered}
    &\gamma_{xy} = \frac{\partial u}{\partial y} + \frac{\partial v}{\partial x} \\
    &\gamma_{yz} = \frac{\partial v}{\partial z} + \frac{\partial w}{\partial y} \\
    &\gamma_{zx} = \frac{\partial w}{\partial x} + \frac{\partial u}{\partial z}
    \end{gathered}
\end{equation}

Substituindo as equações \ref{eq:deformaçãoengenharia} e \ref{eq:angulardef}
na equação \ref{eqn:interpolation}, pode-se expressar as deformações
$\epsilon_{e} = (\epsilon_{x}, \epsilon_{y}, \epsilon_{z}, \gamma_{xy}, \gamma_{yz}, \gamma_{zx})$ da seguinte forma:

\begin{align}\label{eq:deformationalpha}
    \begin{aligned}
        \begin{split}
            \epsilon_{x} &= \alpha_{2}, \epsilon_{y} = \alpha_{7},    \epsilon_{z} = \alpha_{12} 
        \end{split}
        \\
        \begin{split}
             \gamma_{xy} &= \alpha_{3} + \alpha_{6}, \gamma_{yz} = \alpha_{8} + \alpha_{11}, \gamma_{zx} = \alpha_{4} + \alpha_{10}
        \end{split}
    \end{aligned}
\end{align}

Substituindo a equação \ref{eq:deformationalpha} na equação \ref{eq:matrixdisplacement}:

\begin{equation} \label{eq:matrixalphadisplacement}
    \begin{Bmatrix}
        \epsilon_{x} \\
        \epsilon_{y} \\
        \epsilon_{z} \\
        \gamma_{xy} \\
        \gamma_{yz} \\
        \gamma_{zx} \\
    \end{Bmatrix}
    = 
    \begin{bmatrix}
    0 & 1 & 0 & 0 & 0 & 0 & 0 & 0 & 0 & 0 & 0 & 0 \\
    0 & 0 & 0 & 0 & 0 & 0 & 1 & 0 & 0 & 0 & 0 & 0 \\
    0 & 0 & 0 & 0 & 0 & 0 & 0 & 0 & 0 & 0 & 0 & 1 \\

    0 & 0 & 1 & 0 & 0 & 1 & 0 & 0 & 0 & 0 & 0 & 0 \\
    0 & 0 & 0 & 0 & 0 & 0 & 0 & 1 & 0 & 0 & 1 & 0 \\
    0 & 0 & 0 & 1 & 0 & 0 & 0 & 0 & 0 & 1 & 0 & 0 \\
    
    \end{bmatrix}
    \begin{Bmatrix}
        \alpha_{1} \\
        \alpha_{2} \\
        \alpha_{3} \\
        \alpha_{4} \\
        \alpha_{5} \\
        \alpha_{6} \\
        \alpha_{7} \\
        \alpha_{8} \\
        \alpha_{9} \\
        \alpha_{10} \\
        \alpha_{11} \\
        \alpha_{12} \\
    \end{Bmatrix}
    ,
\end{equation}
    
\begin{equation} \label{eq:simplifiedstrain}
    \epsilon_{e} = \pmb{B}\alpha
\end{equation}    

Substituindo \ref{eq:simplifiedstrain} na equação \ref{eq:simplifieddisplacement} obtemos a seguinte relação:

\begin{equation} \label{eq:strain-displacement}
    \epsilon_{e} = \pmb{B}_{e}\pmb{u}_{e},
\end{equation}    

Onde a matriz $\pmb{B}_e = \pmb{B}\pmb{A}^{-1} $ é chamada de matriz deformação-deslocamento.

%Adicionar referências

Segundo a Lei de Hooke, tensão e deformação podem ser escritos da seguinte forma:

\begin{equation} \label{eq:tensaodeformacao}
    \begin{aligned}
        \epsilon_{x} = \frac{\sigma_{x} - \nu(\sigma_{y} + \sigma_{z})}{E},
        \epsilon_{y} = \frac{\sigma_{y} - \nu(\sigma_{z} + \sigma_{x})}{E},
        \epsilon_{z} = \frac{\sigma_{z} - \nu(\sigma_{x} + \sigma_{y})}{E}
    \end{aligned}    \\
    
    \begin{aligned}
        \gamma_{xy} = \frac{\uptau_{xy}}{G},
        \gamma_{yz} = \frac{\uptau_{yz}}{G},
        \gamma_{zx} = \frac{\uptau_{zx}}{G},
    \end{aligned}
\end{equation}

onde $E$, $G$ e $\uptau$ são, respectivamente, o Módulo de Young, Módulo de cisalhamento e Coeficiente de Poisson.
Para materiais homogêneo e isotrópico, o Módulo de Young pode ser escrito como $E = 2G(\uptau + 1)$. Assim, as equações \ref{eq:tensaodeformacao} podem ser escritas da seguinte forma:
